% !TEX program = xelatex

% ==== Part1: 引入latex需要的package,支持不同的需求,如中文、图片 ====
% \documentclass{article}
\documentclass{book}
\usepackage[UTF8]{ctex} % 中文latex支持
\usepackage{graphicx} % 引入图片时需要的包
\usepackage{float}    % 插入图片的时候支持`H`这个位置选项
\usepackage{subcaption} % 插入多张图片到一个figure域中
\usepackage{hyperref}   % 插入超链接、TOC支持链接
\usepackage{amsmath,bm} % 一些数序符号、字符
\usepackage{listings}   % 插入代码块
% ==== 对引入的package配置 ====
\graphicspath{{images/}} % 配置graphicx这个包:指定图片存储的地方
\hypersetup{ % 设置超链接和url的显示样式
    colorlinks=true,
    linkcolor=blue,
    filecolor=magenta,      
    urlcolor=cyan,
    pdftitle={Overleaf Example},
    pdfpagemode=FullScreen,
    }
\urlstyle{same}

% ==== Part2: latex文档格式配置 ====
\let\cleardoublepage\clearpage % 删除chapter之间不想要的空白页面
\newif\ifchinese % 定义一个条件变量: ifchinese, 作为控制条件编译的开关
\chinesetrue % 条件编译的控制开关,注释掉此部分内容,则对应部分不会被现实
\newif\ifchinese % 定义一个条件变量: ifchinese, 作为控制条件编译的开关
\chinesetrue % 条件编译的控制开关,注释掉此部分内容,则对应部分不会被现实

% ==== Part3: latex文档标题部分 ====
\title{线性代数学习笔记}
\author{付杰\thanks{感谢GILBERT STRANG老师的书籍和教学}}
\date{2023年4月15日 至 \today}


% ==== Part4: latex文档正文部分 ====
\begin{document}
\maketitle
\newpage

% \begin{abstract} % 摘要
\chapter*{前言} % 摘要
线性代数是做科研时需要掌握的最重要的数学知识之一。在我认为,在“计算机体系结构”,“数字集成电路设计”的科研过程当中,线性代数出现的频率比“微积分”和“概率论”出现的频率还高。

例如在计算机科学中,很多高性能计算相关的课题因为需要并行计算,因此是跟矩阵计算有关的,很多的算法跟数据结构都可以转化成矩阵的形式,例如图相关问题;在数字通信领域,例如在设计5G基带处理器的时候,处理器内部有很多的加速器用以加速编码、调制、解交织等过程,这些都是计算密集型计算,就会涉及到很多的矩阵乘法、举证求逆的操作;在“数字信号处理“和”深度学习“的学科中,也充斥着很多矩阵计算和优化(optimization)的问题。
因此可以说线性代数在“计算机科学”、“电子科学”和“通信工程”中,扮演者非常非常重要的角色,如果没有基础的线性代数知识,在硕士阶段的科研过程中就会感到很局促、在博士科研过程中就根本无法开展工作。

本笔记主要用于回忆考研过后就遗忘的“线性代数”相关的知识,相当于是重新学习一遍的过程;另外在重学现代的过程中,我也会结合自己现在研究生阶段时遇到的问题、新的学习方法,去以\textbf{一种新的高度和视角}去学习线性代数,尽力去让自己避免“死记硬背”,而是从“理解、实用”的角度去掌握线性代数。

  参考的书籍主要是GILBERT STRANG老师的著作如\href{https://math.mit.edu/~gs/linearalgebra/ila5/indexila5.html}{《Introduction to Linear Algebra, fifth edition》}、《Linear Algebra and Its Applications, fifth edition》。第一章节主要记录线性代数的基础知识,第二章节会结合计算机科学、电子科学和通信技术中遇到的线性代数应用,展示线性代数是如何在科研中发挥作用的。
  \par\textbf{关键字:} 线性代数, 矩阵乘法,线性代数应用

\tableofcontents % 插入目录
\chapter{线性代数基础知识} % (fold)
\label{cha: chapter1}

\section{向量}%
在\textbf{向量}这一节主要介绍如下3部分向量的相关基础知识:
\begin{enumerate}
  \item 线性组合(linear combination)
  \item 向量点积(dot product)
  \item 矩阵和线性方程组$A\boldsymbol{x}=\boldsymbol{b}$
\end{enumerate}
\subsection{线性组合}
  \label{sec:vectorIntro}
  向量的\textit{线性组合}两个内容:\textit{向量乘以常数,向量加上向量}

  \textbf{公式1:}
\begin{equation}c \boldsymbol{v}+d \boldsymbol{w}=c\left[\begin{array}{l}1 \\ 1\end{array}\right]+d\left[\begin{array}{l}2 \\ 3\end{array}\right]=\left[\begin{array}{l}c+2 d \\ c+3 d\end{array}\right]\label{eq: vec1}\end{equation}

  \begin{itemize}
    \item c和d可以取任意值
    \item   
      \begin{equation*}
              c\left[\begin{array}{c}
                1\\1
              \end{array}\right],
              d\left[
                \begin{array}{c}
                2\\3 
                \end{array}
                \right]
      \end{equation*}
      分别表示在二维平面上沿着$(x,y)=(1,1)$的任意长度的直线、沿着$(x,y)=(2,3)$的任意长度的直线
    \item 由于这两条直线\textbf{不重合},因此公式\ref{eq: vec1}可以表示二维平面上所有的点, 即其可以\textbf{张成(span)整个二维平面}。
  \end{itemize}


\textbf{公式2:}
\begin{equation}
  c\left[\begin{array}{c}
  1\\1\\1
\end{array}\right]+
  d\left[\begin{array}{c}
    2\\3\\4
  \end{array}\right]
  \label{eq: vec2}
\end{equation}

  公式\ref{eq: vec2}表示\textit{三维空间里的一个平面},因为:
  \begin{itemize}
    \item 每个向量包含3个元素,表示三维空间里的一条线。一个向量如果有n个元素,那么它就是n维空间的向量。
    \item 公式\ref{eq: vec2}的线性组合表示\textit{三维空间里的两条不重合的直线的组合,其可以张成(span)三维空间里的一个平面}
  \end{itemize}

\textbf{公式3:}
\begin{equation}
  c\left[\begin{array}{c}
  1\\1\\1
\end{array}\right]+
  d\left[\begin{array}{c}
    2\\3\\4
  \end{array}\right]=
  \left[
    \begin{array}{c}
      1\\0\\0
    \end{array}
    \right]=\boldsymbol{b}
  \label{eq: vec3}
\end{equation}

公式\ref{eq: vec3}尝试找到可能的c,d的组合使得等式成立,但是该等式无解。
因为$\boldsymbol{b}$不在公式\ref{eq: vec2}所张成的平面上。


PS: 为了节约文章的空间,后续文档中,列向量$\boldsymbol{v}$采用如下写法
\begin{equation*}
  \boldsymbol{v}=\left[
      \begin{array}{c}
        a\\b\\c
      \end{array}
      \right]=\left(a, b, c\right)
\end{equation*}
请勿因为$\boldsymbol{v}=\left(a, b, c\right)$而把$\boldsymbol{v}$认为是行向量,$\boldsymbol{v}=\left[a, b, c\right]$才是行向量的表示形式。

\subsection{长度 \& 点积(dot product)}
\textbf{向量相乘}
\begin{equation}
  \boldsymbol{v}=(a, b, c), \boldsymbol{u}=(x, y, z), \boldsymbol{v}\cdot \boldsymbol{u}=ax+by+cz 
  \label{eq: dotProduct1}
\end{equation}

\textbf{向量之间的角度}
\begin{equation}
  \boldsymbol{v}=(a, b, c), \boldsymbol{u}=(x, y, z), \cos{\theta}=\frac{v\cdot u}{\lVert u \rVert \cdot \lVert u \rVert }
  \label{eq: dotProduct2}
\end{equation}
\begin{itemize}
  \item 向量平行(parallel)的时候,点积的值最大
  \item 向量垂直(perpendicular)的时候,点积的值最小且为0
\end{itemize}

\textbf{柯西不等式\&三角不等式}
\begin{equation}
  \begin{aligned}
    &\text{柯西不等式: }\vert v \cdot u \vert \leq  \lVert v \rVert \lVert u \rVert \\
    &\text{三角不等式: } \vert v+u \vert \leq \lVert v \rVert + \lVert u \rVert \\
  \end{aligned}
  \label{eq: unequal}
\end{equation}

\section{矩阵Matrix}
将公式\ref{eq: vec3}中的$\boldsymbol{c, d}$换成$\boldsymbol{x_1, x_2}$,可以得到

\begin{equation}
   x_1\left[
     \begin{array}{c}
      1\\1\\1
     \end{array}\right] 
   +x_2\left[
     \begin{array}{c}
      2\\3\\4
     \end{array}\right]=
   \left[\begin{array}{c c}
     1 & 2\\
     1 & 3\\
     1 & 4\\
   \end{array}\right]
   \left[ 
   \begin{array}{c}
     x_1\\
     x_2
   \end{array}\right]=\boldsymbol{Ax}
  \label{eq: matrix1}
\end{equation}

\subsection{矩阵乘向量的两种理解}

在式\ref{eq: matrix1}的右边添加目标列向量$\boldsymbol{b}$,尝试找到可能的x,满足公式\ref{eq: matrix2},则引出了矩阵求解的问题。
\begin{equation}
  \boldsymbol{Ax}=\boldsymbol{b}
  \label{eq: matrix2}
\end{equation}

矩阵求解的问题主要有两个:\textit{是否有解}? 和 \textit{解是什么}? 
对矩阵乘法$\boldsymbol{Ax}$的理解如下:
\begin{enumerate}
  \item \textbf{Column Picture(从列向量线性组合的角度)}: 矩阵A中的每一列都是一个\textit{列向量, $\boldsymbol{Ax}$}表示所有A中列向量的\textit{线性组合} 

    $\boldsymbol{Ax}=0$表示所有列向量的线性组合等于0。当$\boldsymbol{x}$中元素全部为0的时候,一定会成立;当$\boldsymbol{x}$中元素不全部为0的时候,$\boldsymbol{A}$的列向量必须相关才可以成立。
        \begin{equation}
          \boldsymbol{Ax}=\left[\boldsymbol{a_1, a_2}, \cdots \boldsymbol{a_n}\right]\cdot\left[
            \begin{array}{c}
              x_1\\x_2 \vdots \\x_n
            \end{array}\right]=x_1a_1+x_2a_2+\cdots+x_na_x
          \label{eq: columnPic}
        \end{equation}
      \item \textbf{Row Picture(从几何相交的角度)}:  矩阵A中每一行都是一个行向量, $\boldsymbol{Ax}$如果有m行,则表示有m个等式,每一个等式都是一个平面,表明这m个平面相交于$\boldsymbol{x}$。
        $\boldsymbol{Ax}=\boldsymbol{0}$表示这m个平面相交于原点(origin)。
        \begin{equation}
          \boldsymbol{Ax}=\left[
            \begin{array}{c}
              row_1 \\ 
              row_2 \\ 
              \vdots\\ 
              row_n
            \end{array}
            \right]\cdot \boldsymbol{x}=
          \left[
            \begin{array}{c}
              row_1\cdot\boldsymbol{x}\\ 
              row_2\cdot\boldsymbol{x} \\ 
              \vdots\\ 
              row_n\cdot\boldsymbol{x}
            \end{array}
            \right]
          \label{eq: rowPic}
        \end{equation}
\end{enumerate}


\subsection{矩阵乘法有解条件}

\begin{enumerate}
  \item \textbf{几何理解:}b在A所有列向量张成的空间里
\end{enumerate}

\subsection{矩阵乘法有唯一解条件}
\begin{enumerate}
  \item 矩阵A\textbf{可逆}: $x=A^{-1}b$ 
  \item 矩阵A所有的列向量\textbf{独立}
  \item \textbf{几何理解:}矩阵A中不存在某个列向量$col_k$,存在于其余列向量张成的空间里
\end{enumerate}

\section{矩阵乘矩阵}
矩阵乘矩阵$AB=C$有两种理解: 
\begin{enumerate}
  \item 矩阵A的行乘以矩阵B的列,得到目标矩阵的一个元素$C_ij$;逐一计算得到所有的C的元素
    \begin{equation*}
      \begin{aligned}
        AB&=\left[\begin{array}{c c}
          1 & 1\\ 
          2 & 2 
        \end{array}\right]\left[
          \begin{array}{c c}
          1 & 2\\ 
          3 & 4 
        \end{array}
          \right]\\
        &=\left[
          \begin{array}{c c}
          4 & x\\ 
          x & x 
        \end{array}
          \right]
        =\left[
          \begin{array}{c c}
          4 & 6\\ 
          x & x 
        \end{array}
          \right]\\
        &=\left[
          \begin{array}{c c}
          4 & 6\\ 
          8 & x 
        \end{array}
          \right]
        =\left[
          \begin{array}{c c}
          4 & 6\\ 
          8 & 12 
        \end{array}
          \right]
      \end{aligned}
    \end{equation*}
  \item 矩阵A的列乘以矩阵B的行,得到一个矩阵;一共可以得到多个矩阵,然后这些矩阵再相加
    \begin{equation*}
      \begin{aligned}
        AB&=\left[\begin{array}{c c}
          1 & 1\\ 
          2 & 2 
        \end{array}\right]\left[
          \begin{array}{c c}
          1 & 2\\ 
          3 & 4 
        \end{array}
          \right]\\
        &=\left[
          \begin{array}{c c}
            1 & 2\\ 
            2 & 4
          \end{array}
          \right]+
        \left[
          \begin{array}{c c}
            3 & 4\\ 
            6 & 8
          \end{array}
          \right]\\
        &=\left[
          \begin{array}{c c}
            4 & 6\\ 
            8 & 12 
          \end{array}
          \right]
      \end{aligned}
    \end{equation*}
\end{enumerate}


在解矩阵方程$\boldsymbol{Ax}=\boldsymbol{b}$的时候,我们需要做\textbf{row elimination}对矩阵A进行化简得到上三角矩阵。
化简的这个过程可以用矩阵乘法来表示。

\textbf{消除矩阵(elimination Matrix) $E_{ij}$}
\begin{equation*}
  \begin{aligned}
    \boldsymbol{Ax}&=\boldsymbol{b} \rightarrow \boldsymbol{E}Ax=\boldsymbol{E}b\\
    \text{if } \boldsymbol{E}&=
    \left[
      \begin{array}{c c c}
        1 & 0 & 0 \\ 
        -l & 1 & 0 \\ 
        0 & 0 & 1
      \end{array}
    \right]\text{then :}\\
  Eb&=\left[\begin{array}{c}
    b_1\\ b_2-b_1 \\ b_3 
  \end{array}\right]\\ 
    EAx&=\left[
      \begin{array}{c}
      row_1\\ row_2-row_1 \\row_3
      \end{array}\right]x
  \end{aligned}
  \label{eq: matrix3}
\end{equation*}

\textbf{行交换矩阵(row exchange Matrix) $P_{ij}$}
\begin{equation*}
  \begin{aligned}
    \boldsymbol{Ax}&=\boldsymbol{b} \rightarrow \boldsymbol{P}Ax=\boldsymbol{P}b\\
    \text{if } \boldsymbol{P}&=
    \left[
      \begin{array}{c c c}
        1 & 0 & 0 \\ 
        0 & 0 & 1 \\ 
        0 & 1 & 0
      \end{array}
    \right]\text{then :}\\
  Pb&=\left[\begin{array}{c}
    b_1\\ b_3 \\ b_2 
  \end{array}\right]\\ 
    PAx&=\left[
      \begin{array}{c}
      row_1\\ row_3\\row_2
      \end{array}\right]x
  \end{aligned}
  \label{eq: matrix4}
\end{equation*}

\begin{enumerate}
  \item $\boldsymbol{EA}$可以看成E作用于A的每一个列向量,其效果跟$\boldsymbol{Eb}$是一样的。
  \item 矩阵乘在A的左边($\boldsymbol{EA}$),则会对A的每一个行向量起作用;\\
    矩阵乘在A的右边($\boldsymbol{AE}$),则会对A的每一个列向量起作用
\end{enumerate}

\section{逆矩阵$A^{-1}$}
\begin{enumerate}
  \item 定义:
    \begin{equation}
      AA^{-1}=A^{-1}A=I
      \label{eq: inverse}
    \end{equation}
  \item 判断条件:
    \begin{enumerate}
      \item 快速的判断方法: 对A进行行化简,最后A有n个povit,否则会违反Gauss-Jordan elimination\\ 
        对角主导矩阵(diagonally dominant matrix): 矩阵对角线元素,比该行所有元素之和还大
      \item 计算A的行列式, $\vert A\vert\neq 0$, 否则会违反公式\ref{eq: inverse}
      \item 解矩阵方程$Ax=0, x=0$是唯一解, 否则$x=A^{-1}0\neq 0$自相矛盾
      \item 对角矩阵,对角线元素不存在0, 否则会违反判断条件(b)
    \end{enumerate}
    不可逆矩阵称为(singular matrix)
  \item 应用
    \begin{enumerate}
      \item $(AB)^{-1}=B^{-1}A^{-1}$ 
      \item $Ax=b\rightarrow x=A^{-1}b$
      \item A沿着对角线对称则$A^{-1}$也沿着对角线对称;\\ 
            A是稀疏矩阵,$A^{-1}$很有可能是稠密矩阵;\\ 
            A是上三角矩阵,$A^{-1}$也是上三角矩阵
    \end{enumerate}
  \item 求逆矩阵$A^{-1}$的方法:
    \begin{enumerate}
      \item Gauss-Jordan elimination: $\left[A \; I\right]\rightarrow [I \: A^{-1}]$
    \end{enumerate}
\end{enumerate}
\section{矩阵分解}
\subsection{LU分解}
\begin{equation}
  \begin{aligned}
    A&=LU\\ 
    A&=LDU
  \end{aligned}
  \label{eq: LU1}
\end{equation}
在做Gauss-Jordan分解得到上三角矩阵U的时候,自然就完成了对矩阵A的LU分解。
其中U是上三角矩阵,L是对A做的行变换的逆矩阵\\ 
进一步将U转化成对角线为1的矩阵: $U=DU$, eq:
\begin{equation*}
  \begin{aligned}
    A&= 0
  \end{aligned}
  \label{eq: LU2}
\end{equation*}

\begin{enumerate}
  \item $A=LU$的计算复杂度: $O(n^3/3)$
  \item LU分解的作用:更方便地求解x, $Ax=b \rightarrow Lc= b, Ux=c$
\end{enumerate}

\subsection{$LDL^T$分解}
如果待做LU分解的矩阵A是\textbf{对称矩阵},则按照LU分解的方法可以做到$LDL^T$分解\\
$A=LDU=LDL^T$, $LDL^T$分解的计算复杂度只有$O(n^3/6)$,是LU分解的一半 
\begin{equation*}
 \begin{aligned}
 S&=\left[
  \begin{array}{c c c}
    1 & 4 &5\\ 
    4 & 2 &6\\ 
    5 & 6 &3 
  \end{array}
 \right]\rightarrow
   \left[
    \begin{array}{c c c}
      1 & 4 &5\\ 
      0 & -14 & -14\\ 
      0 & -14 & -22
    \end{array}
   \right]\rightarrow
    \left[
      \begin{array}{c c c}
      1 & 4 &5\\ 
      0 & -14 & -14\\ 
      0 & 0 & -8 
      \end{array}
    \right]=U\\ 
  D&=\left[
    \begin{array}{c c c }
      1 & 0 & 0\\ 
      0 & -14&0\\ 
      0 & 0 &-8
    \end{array}
  \right], 
  L=\left[
    \begin{array}{c c c }
      1 & 0 & 0\\ 
      4 & 1 & 0\\ 
      5 & 1 & 1
    \end{array}
  \right]
 \end{aligned} 
\end{equation*}
D和L可以直接由U计算得到,简化了LU分解的复杂度

\section{转置(Transpose)}
\begin{enumerate}
  \item inner product: $\boldsymbol{x\cdot y}=\boldsymbol{x}^T\boldsymbol{y}$,结果是一个数\\ 
        outer product: $\boldsymbol{xy^T}$ 是n x n 矩阵
  \item $(A+B)^T=A^T+B^T$\\ 
        $(AB)^T=B^TA^T$\\
        $(A^{-1})^T=(A^T)^{-1}$\\ 
        A下三角:$A^{-1}$下三角、$A^T$上三角\\
  \item $(Ax)^Ty=x^TA^Ty$
  \item $A^TA\neq AA^T$恒为对称矩阵
\end{enumerate}
\textbf{Perputation Matrix}: 
P是矩阵I任意打乱行与行之间的顺序得到的矩阵,一共有n!中P矩阵, 
$P^{-1}=P^T$\\ 
P乘以矩阵A,会把A每行的顺序按照P每行的顺序改变,例如:
\begin{equation*}
 \begin{aligned}
  P=\left[
    \begin{array}{c c c }
      0 & 1 & 0\\ 
      0 & 0 & 1\\ 
      1 & 0 & 0
    \end{array}
  \right],  
  A=\left[
    \begin{array}{c }
     row_1\\ 
     row_2\\ 
     row_3
    \end{array}
  \right]\rightarrow
  PA=\left[
    \begin{array}{c }
     row_2\\ 
     row_3\\ 
     row_1
    \end{array}
  \right]
 \end{aligned} 
\end{equation*}
在做LU分解的时候,如果矩阵A的pivot不在正确的位置,可以先用矩阵P乘以A,在对PA做LU分解
\begin{equation*}
 A=LU \rightarrow PA=L'U'
\end{equation*}

% \newpage 
% \bibliography{ref} % 参考文献源,存储所有的参考文献
% \bibliographystyle{IEEEtran} % latex饮用参考文献时的格式

\end{document}

