% !TEX program = xelatex

% ==== Part1: 引入latex需要的package,支持不同的需求,如中文、图片 ====
\documentclass{article}
\usepackage[UTF8]{ctex} % 中文latex支持
\usepackage{graphicx} % 引入图片时需要的包
\usepackage{float}    % 插入图片的时候支持`H`这个位置选项
\usepackage{subcaption} % 插入多张图片到一个figure域中
\usepackage{hyperref}   % 插入超链接、TOC支持链接
\usepackage{amsmath,bm} % 一些数序符号、字符
\usepackage{listings}   % 插入代码块
% ==== 对引入的package配置 ====
\graphicspath{{images/}} % 配置graphicx这个包:指定图片存储的地方
\hypersetup{ % 设置超链接和url的显示样式
    colorlinks=true,
    linkcolor=blue,
    filecolor=magenta,      
    urlcolor=cyan,
    pdftitle={Overleaf Example},
    pdfpagemode=FullScreen,
    }
\urlstyle{same}

% ==== Part2: latex文档格式配置 ====
\newif\ifchinese % 定义一个条件变量: ifchinese, 作为控制条件编译的开关
\chinesetrue % 条件编译的控制开关,注释掉此部分内容,则对应部分不会被现实
\newif\ifchinese % 定义一个条件变量: ifchinese, 作为控制条件编译的开关
\chinesetrue % 条件编译的控制开关,注释掉此部分内容,则对应部分不会被现实

% ==== Part3: latex文档标题部分 ====
\title{基于RISC-V的嵌入式低功耗MCU设计}
\author{付杰}
\date{2023 年 4 月 19日 to \today}


% ==== Part4: latex文档正文部分 ====
\begin{document}
\maketitle
\newpage

\begin{abstract} % 摘要
  RISC-V指令集低功耗嵌入式实时响应MCU,五级流水线设计
  \par\textbf{关键字:} RISC-V, MCU, 低功耗
\end{abstract}
\newpage

\tableofcontents % 插入目录
\newpage


\section{MCU整体特点及实现方案}
\subsection{流水线}
\subsubsection{IF}
\begin{enumerate}
  \item \textbf{静态分支预测}\cite{riscv1}:
    \begin{itemize}
      \item B-Type: FNBT
      \item JAL: 直接做静态预测,直接跳转
      \item JALR:只对x0, x1做硬件加速;其余情况下不对JALR做分支预测
    \end{itemize}
\end{enumerate}
\subsubsection{ID}
\subsubsection{EXE}
\subsubsection{MEM}
\subsubsection{WB}
\subsection{低功耗}
\begin{enumerate}
  \item \textbf{逻辑门控}\cite{riscv1}:加法器、移位器等运算资源内 部包含大量逻辑门,若在空闲时放任其翻转将造成大量功耗开销。\\
   使用逻辑门控的方法,在不使用运算资源时强制其输入为0,从而避免上述问题
\end{enumerate}
\subsection{加速器控制}




\newpage
\bibliography{mcuRef} % 参考文献源,存储所有的参考文献
\bibliographystyle{IEEEtran} % latex饮用参考文献时的格式

\end{document}

